%\input{./Anexos/PortadaETSII} % Por ej.,: Otras portadas
%\includepdf{fichero_portada.pdf} % Incluso como fichero PDF

% -------------------------------------------------------------------------
% PORTADA PRAL. (1)
% -------------------------------------------------------------------------

% \portadaOld % Portada pral.
\portadaNew

% -------------------------------------------------------------------------
% PORTADA INTERIOR (2)
% -------------------------------------------------------------------------
\portadaInt %Portada interior


% -------------------------
% CRÉDITOS
% -------------------------
\begin{creditos}[\titulo] % Editar a conveniencia (se pasa el título deseado)
The author may choose the license type they wish. This document is distributed under license CC BY-NC-SA 4.0. The full text of the license is available at \url{https://creativecommons.org/licenses/by-nc-sa/4.0/}. The copy and distribution of this work is permitted in all parts of the world, without royalties and in any medium, as long as this notice is preserved. In addition, permission is granted to copy and distribute translations of this book from the English original to another language, provided that the copyright notice and this permission notice are preserved in all copies.

\noindent \includegraphics[width=0.15\linewidth]{by-nc-sa}

% Para citar este plantilla empleando bibtex puedes emplear el registro siguiente:
%@www{salidoTFG,
%  author       = {Jesús Salido},
%  title        = {Plantilla guía de TFG para la ESI-UCLM},
%  year         = {2019},
%  editor       = {GitHub},
%  organization = {Universidad de Castilla-La Mancha},
%  url          = {https://github.com/JesusSalido/TFG_ESI_UCLM},
%  doi          = {10.5281/zenodo.4574562}
%}
\end{creditos}


% -------------------------
% CALIFICACIÓN DEL TRIBUNAL  (No necesario en versión electrónica)
% -------------------------
%\tribunal % Página opcional para calificación 


% -------------------------
% DEDICATORIA 
% -------------------------
\begin{dedicatoria} % (no confundir con los agradecimientos)
\emph{To my family, close friends, fellow colleagues, classmates and professors \\ % A alguien muy especial
For their patience and for accompanying me and showing me the way through this exciting journey.}
\end{dedicatoria}

\pagestyle{plain}	% Páginas sólo con numeración inferior al pie


% -------------------------
% RESÚMENES:
% -------------------------

% EDITAR: Resumen en idioma alternativo default=english (máx. 1 pág.)
%---
\begin{resumenAlt}[english]{\tituloCortoAlt} 
% Se pasa el idioma (opcional) y título

\emph{<<What>>}

Artificial Intelligence is on the rise, and many companies are striving to create a wide-ranging variety of models to help themselves and their customers in an equally wide range of specific tasks. This inevitably translates into multiple model trainings being performed a year and, in many cases, many datasets being created or modified in the same time interval. 
Such is the case of UBOTICA Technologies: a Space:AI company that puts faith in Computation \emph{on the edge} to deliver AI solutions integrated in embedded systems incorporated in spatial modules, which come with limited space for storing data and computing power. The development of these solutions requires multiple training and deployment iterations,
which over the years led to the current tangled mess of untraceable datasets and models, which makes it difficult to search for a specific AI training configuration. 


\emph{<<How>>}

MADTrack is a distributed configuration management system with the purpose of putting order to the configurational chaos previously mentioned. It will store, track and manage all changes within datasets and AI model configurations. The main 
restrictions over the development of the system are the limited storage space of the company for this resources (the management of the evolution of datasets and models has to be done efficiently), the distributed nature of the environment where the items 
are stored and managed and the need for the system to be integrated in a greater processing workflow. The development of the system will be divided in a series of prototypes with an iterative and incremental approach, following continuous testing policies.



\emph{<<Conclusion>>}

The resulting system will consist of a distributed system following the client-server application, with a local or remote server attending requests from multiple clients sending requests by means of a software wrapper library which in turn will interact with other open-source technologies.

\end{resumenAlt}

% EDITAR: Resumen en idioma pral. default=spanish (máx. 1 pág.) 
\begin{resumenPral}[spanish]{\titulo} % Se pasa el idioma (opcional) y título

\emph{<<Que>>}

La Inteligencia Artificial está en ascenso, y muchas empresas están trabajando en la elaboración de una inmensa variedad de modelos que las ayuden tanto a ellas 
como a sus clientes a realizar una variedad de tareas igualmente amplia. Esto provoca que se realicen muchas entrenamientos de modelos al año y, en muchos casos, 
se creen o modifiquen muchos conjuntos de datos en el mismo intervalo de tiempo. Este es el caso de UBOTICA Tecnologías: una empresa de Space:AI que apuesta por la
computación \emph{on the edge} para ofrecer soluciones de Inteligencia Artificial integradas en sistemas empotrados en módulos espaciales, que tienen un espacio de 
almacenaje y poder computacional limitado. El desarrollo de estas soluciones requiere de varias iteraciones de entrenamiento y despliegue, lo que ha llevado a un desorden
caótico de conjuntos de datos y modelos difícilmente identificables, lo que dificulta la búsqueda de un modelo o configuración de entrenamiento específica. 

\emph{<<Como>>}

MADTrack es un sistema de gestión de configuración distribuido que tiene el propósito de ponerle orden al caos configuracional previamente mencionado. El sistema almacenará, 
rastreará y gestionará todas las modificaciones de conjuntos de datos y configuraciones de modelos IA.Las principales
restricciones que tiene el desarrollo del sistema son el espacio de almacenamiento limitado de la empresa para estos recursos (la evolución de conjuntos de datos y modelos debe ser 
registrada de manera eficiente), la naturaleza distribuida del entorno donde se almacenan y gestionan los elementos y la necesidad de la integración del sistema en un gran flujo de
procesamiento. El desarrollo del sistema se dividirá en una serie de prototipos con un enfoque iterativo y incremental, siguiendo políticas de pruebas continuas.


\emph{<<Conclusiones>>}

El sistema resultante consistirá en un sistema distribuido que sigue la arquitectura cliente-servidor, con un servidor local o remoto atendiendo las solicitudes de varios clientes enviando 
solicitudes a través de una biblioteca Software del tipo Wrapper que a su vez se interactúe con otras tecnologías de software abiertas. 

\end{resumenPral}




% Ajuste al idioma pral.
\ifbool{ESI@spanish}{\selectlanguage{english}}{\selectlanguage{spanish}}


% -------------------------
% AGRADECIMIENTOS (máx. recomendable: 1 pág.)
% -------------------------
\auxchapter{Agradecimientos} % Editar a conveniencia
Aunque es un apartado opcional, haremos bueno el refrán \emph{<<es de bien nacidos, ser agradecidos>>} si empleamos este espacio como un medio para agradecer a todos los que, de un modo u otro, han hecho posible que el trabajo realizado \emph{llegue a buen puerto}. Esta sección es ideal para agradecer a directores, profesores, mentores, familiares, compañeros, amigos, etc. 

Estos agradecimientos pueden ser tan personales como desees e incluir anécdotas y chascarrillos, pero recuerda que \emph{no deberían ocupar más de una página}.

\firma % Nombre, lugar y año (automático, no cambies)


% -------------------------
% -NOTACIÓN: Lista de símbolos con significado especial.
% -------------------------
\auxchapter{Notación y acrónimos}
\section*{Notacion}
(Texto aclaratorio \emph{-suprime-}). Ejemplo de lista con notación (o nomenclatura) empleada en la memoria del TFG. Debes editarla según las necesidades de tu trabajo intenta que sea informativa y evita que incorpore información obvia.\footnote{Se incluye únicamente con propósito de ilustración, ya que el documento no emplea la notación aquí mostrada.}

\begin{tabular}{r r p{0.8\linewidth}}
$A, B, C, D$	& : & Variables lógicas \\
$f, g, h$		& :	& Funciones lógicas \\
$\cdot$			& : & Producto lógico (AND), a menudo se omitirá como en $A 
B$ en lugar de $A \cdot B$\\
$+$				& : & Suma aritmética o lógica (OR) dependiendo del 
contexto\\
$\oplus$		& : & OR exclusivo (XOR)\\
$\overline{A}$ o ${A}'$	& : & Operador NOT o negación
\end{tabular}

\section*{Lista de acrónimos}
% OJO: Esta lista debería estar ordenada alfabeticamente (hacer de modo manual).
(Texto aclaratorio \emph{-suprime-}). Ejemplo de lista \emph{ordenada alfabéticamente} con los acrónimos empleados en el texto. Se pueden omitir aquellos acrónimos que son reconocidos en el contexto académico (p.~ej., PhD), aunque aquí se han incluido a efectos ilustrativos.

\printacronyms 


% -------------------------
% ÍNDICES: Elimina los innecesarios.
% -------------------------
\idxGral
\idxFiguras
\idxTablas
\idxListados
\idxAlgoritmos
%---


