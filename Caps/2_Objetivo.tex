\chapter{Objective}
\label{cap:Objective}

The main aim of this chapter is to define the main objective of the project, and also enumerate and detail the specific objectives that will be necessary to fulfill so as to
achieve this main objective. The definition of the objective and the specific objectives has the pupose of providing a better explanation of the work that will be carried out
within the scope of the project, and to provide an indicator of its good progress.

\section{Main Objective}

The main aim of the project is to produce a system capable to track the configuration of datasets and \acrshort{AI} models, and is easy to integrate in the internal workflows
and pipelines of the end user (UBOTICA Technologies). The system will also be characterized by its distributed nature, its scalability (it will
make efficient use of the resources, so that an increase on the resources or number of users will make a minimal impact on the system's performance), security (the system must
be prepared for possible attacks, specially from injection and buffer overflow attacks) and robustness (Upon the case of minor failures, the system must not go down and provide
adequate and meaningful logging).

The system will provide user-friendly mechanisms for accessing the dataset and model configuration database, so the users can make operations on this configuration directly from
their codespaces.

\section{Specific Objectives}

The aforementioned main objective can be divided into a set of partial objectives, also referred to as subobjectives. This final project can be divided into -- subobjectives
that will mark the development progress of the project, which could be in turn considered as finished when all of the subobjectives have been completed, and the resulting system
satisfies the specifications of the main objective.

\begin{itemize}
	\item \emph{Development and deployment plannning of a server able to track the configuration of \acrshort{AI} models and datasets.}
	
	A planning will be made regarding the deployment details and the necessary infrastructure to be able to hold the Tracking server that will satisfy the requests from the
	other components of the system. These details involve the necessary hardware requirements (Memory, CPU cores, network configuration, available ports ) and 
	software requirements (dependencies and entrypoint scripts) that will be used to design and develop the tracking server, which will be deployed in the future inside of the
	company's intranet infrastructure. It is also necessary to specify how this server will interact with the rest of components of the system and when should it be 
	deployed so as to ensure its well-functioning.

	\item \emph{Development of a library module that manages the configuration management of datasets. Registering changes on their contents.}
	
	Aside from the tracking server, multiple library components will be necessary to manage the configuration of datasets and models. Some of these components will be developed
	under a module that will handle issues regarding the configuration management of datasets. These components will focus on providing code mechanisms that enable the integration
	of new datasets into the system, as well as providing the necessary means to bring the datasets in a specific evolutionary stage to the users.

	\item \emph{Development of a library module that manages the configuration management of AI models, and facilitates the search of models according to their performance.}
	
	Other components of the library will be gathered within a module focused on managing the configuration of Artificial Intelligence models. These components will interact
	with the tracking server in order to track the parameters and metrics produced by the experimental runs of the models performed by the company, and register the final models
	and any other file meaningful to these inside a database.
\end{itemize}