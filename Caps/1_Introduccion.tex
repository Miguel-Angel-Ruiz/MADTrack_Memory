\chapter{Introduction}
\label{cap:Introduction}

This chapter holds the aim is to give a  general introduction to what motivated the development of this work, its objectives, and an
overview of the context where the work is going to be developed. Furthermore, a study of some existing tools that solve the problem in
some form or another will be also included.
\section{Motivation}

It is of common knowledge that Artificial Intelligence has gained significant importance in the 2020s. The development of tools such as the GPT models
have headed a revolution in the way humans solve both simple and complex tasks. A revolution that was made possible with the intervention of multiple
companies and organisations, and a considerable amount of money and time invested in the development of these models \cite{AIRise}.

UBOTICA Technologies is a pioneer company that develops \acrshort{AI} \acrfull{CV} solutions. This means that, as a company, they use Artificial Intelligence
techniques to extract information from images. These solutions are then integrated in embedded systems with limited capabilities, and that are part of a 
bigger, more complex system. The domains where these solutions are used are mainly in the space industry with their new CogniSAT-6 project \cite{UBOTICACS6}, 
and recently even in the culinary industry. The development of the various solutions of the company requires multiple iterations where the models are trained,
validated and tested, either in existing datasets, or in new ones. Moreover, the CogniSAT-6 system has the capability for creating new datasets out of self-taken
images, which may be periodically added to the existing datasets, or even used to replace the existing ones.

This continuous rise of the available models and datasets, paired with a lack of a real control protocol over the new and improved versions of an AI model or dataset,
has led often to chaotic situations where an abnormal amount of time is taken on looking for the desired dataset, and accessing the necessary training configuration
that produced an specific result on a model.

The aim of MADTrack is to develop a system that establishes a real configuration management basis for these datasets and AI models. A system that can integrate
both new and existing datasets and models in a distributed, remote environment, and that will make the best use of the available resources the company dedicates
to this management.

\section{Contexto disciplinar y tecnológico}
También se puede denominar \emph{<<Antecedentes>>} o \emph{<<Estado del arte>>} cuando se trata de comentar trabajos relacionados que han abordado la cuestión u objetivo planteado. En esta sección se debería introducir el \emph{contexto disciplinar y tecnológico} en el que se desarrolla el trabajo de modo que ayude a entender con facilidad su ámbito y alcance. Puesto que un TFG no tiene que ser necesariamente un trabajo con aportes novedosos u originales, solo es necesario la inclusión de \emph{estado del arte} cuando este contribuya a aclarar aspectos clave del TFG o se desee justificar la originalidad del trabajo realizado.

Para redactar un trabajo académico de modo efectivo se recomienda seguir pautas para obtener un resultado final claro y de lectura fácil, como las expuestas en el blog de Leonor Zozaya~\cite{zozaya17} o el apartado de comunicación eficaz del Departamento de Lengua y Estilo de la UOC~\cite{uoc}.

A la hora de redactar el texto se debe poner especial atención para evitar el plagio\footnote{\url{https://www.uclm.es/areas/biblioteca/encuentra-informacion/perfiles/alumno/antiplagio}} respetando los derechos de propiedad intelectual~\cite{uc3m21} y el uso lícito de gráficos e imágenes procedentes de Internet que no sean de elaboración propia. En este sentido se sugiere la consulta del manual de la Universidad de Cantabria~\cite{unican18}. Dicho documento explica de modo conciso cómo incluir imágenes en un trabajo académico de modo apropiado.

Recuerda que el uso de herramientas de IA (Inteligencia Artificial) está permitida para la revisión de textos y conseguir una redacción final de mayor claridad.\footnote{\url{https://biblioteca.uclm.es/Investiga/Apoyoinvestigacion/IAeninvestigacion}} Por supuesto, también para conseguir emplear \LaTeX{} de modo correcto. Sin embargo, la inclusión directa de los textos generados mediante dichas herramientas es sancionable, ya que su autoría no puede ser atribuida a quien firma el TFG. La tabla~\ref{tab:ia} resume los comportamientos que deberías evitar durante la realización de tu TFG.

\begin{table}[H]%
	\centering
	\caption{Usos ilícitos de la IA en el TFG}
	\label{tab:ia}
	\begin{tabular}{ | p{0.3\linewidth} | p{0.3\linewidth} | p{0.3\linewidth} |}
		\hline
		\textbf{Uso de la IA} & \textbf{Descripción} & \textbf{Riesgo} \\
		\hline
		Generación completa o parcial de textos para la memoria.&
		Presentar como propio un texto obtenido casi en su totalidad por una IA. &
		\textbf{Plagio académico}: el autor no es quien firma el texto.\\
		\hline
		Evitar el trabajo intelectual o de análisis. &
		Usar IA para hacer razonamientos, interpretaciones o críticas sin comprensión real. &
		Viola los principios de \textbf{evaluación auténtica} y aprendizaje significativo. \\
		\hline
		Falsificación de datos. & Generar datos simulados o inexistentes para experimentos, encuestas o estadísticas. &
		Constituye \textbf{fraude académico}.\\
		\hline
		Traducción automática sin revisión. &
		Entregar traducciones automáticas sin control de calidad. &
		Puede derivar en \textbf{errores conceptuales}.\\
		\hline
	\end{tabular}
\end{table}


\section{Estructura del documento}
Este capítulo suele finalizar con una sección en la que se indica la estructura (capítulos) del documento y el contenido de cada una de las partes en que se divide. Veamos a continuación cómo sería esta sección para este documento en concreto.

A lo largo de los capítulos que componen esta guía se muestran ejemplos de elementos de organización del texto en un documento preparado con \LaTeX{}. Los ejemplos mencionados, así como los recogidos en obras de referencia, se pueden emplear para adaptar este documento a las necesidades particulares  \cite{lamport94,wikibookLaTex10}. Entre las obras de consulta disponibles sobre \LaTeX{} se recomienda el uso de las obras gratuitas en español~\cite{oetiker14,borbon21} y las guías disponibles en la página web de \href{https://es.overleaf.com/learn}{Overleaf} (en inglés).

En esta plantilla de TFG se ha optado por seguir la estructura orientativa que puede tener un TFG en la \mbox{ESI-UCLM}. Esta estructura consta de los capítulos siguientes:

\begin{enumerate}
\item \textbf{Introducción}. Donde se trata la motivación y la pertinencia del trabajo. Prosigue con el enunciado conciso del propósito del trabajo y la descripción de su contexto disciplinar y técnico.

\item \textbf{Objetivo}. En el que se detalla el alcance del objetivo general y los secundarios del trabajo.

\item \textbf{Plan de gestión del trabajo}. Describe la estrategia para abordar las distintas fases del trabajo. 

\item \textbf{Desarrollo y resultados}. En este capítulo se explica cómo se han llevado a cabo las fases del trabajo cumpliendo el plan previsto y enumerando los resultados obtenidos.

\item \textbf{Conclusiones}. En el que se realiza una discusión sobre los resultados obtenidos y cómo estos satisfacen los objetivos planteados. Además, se justifica la aplicación al TFG de las competencias adquiridas durante los estudios de grado. También, puede incluir una explicación sobre los trabajos derivados y futuros si estos están planificados o iniciados, así como una breve valoración personal.

\item \textbf{Bibliografía}. Lista de las referencias bibliográficas que se hayan citado en el texto. Recuerda que no debes incluir fuentes de información relacionada que no hayas citado explícitamente en la memoria.

\item \textbf{Anexos}. Contenidos auxiliares que complementan del trabajo, como manuales de uso, diagramas, figuras, tablas, listados de código, etcétera.
\end{enumerate}









